\documentclass[]{article}
\usepackage{lmodern}
\usepackage{amssymb,amsmath}
\usepackage{ifxetex,ifluatex}
\usepackage{fixltx2e} % provides \textsubscript
\ifnum 0\ifxetex 1\fi\ifluatex 1\fi=0 % if pdftex
  \usepackage[T1]{fontenc}
  \usepackage[utf8]{inputenc}
\else % if luatex or xelatex
  \ifxetex
    \usepackage{mathspec}
  \else
    \usepackage{fontspec}
  \fi
  \defaultfontfeatures{Ligatures=TeX,Scale=MatchLowercase}
  \newcommand{\euro}{€}
\fi
% use upquote if available, for straight quotes in verbatim environments
\IfFileExists{upquote.sty}{\usepackage{upquote}}{}
% use microtype if available
\IfFileExists{microtype.sty}{%
\usepackage{microtype}
\UseMicrotypeSet[protrusion]{basicmath} % disable protrusion for tt fonts
}{}
\usepackage[margin=1in]{geometry}
\usepackage{hyperref}
\PassOptionsToPackage{usenames,dvipsnames}{color} % color is loaded by hyperref
\hypersetup{unicode=true,
            pdftitle={GVPT392(849): Introduction to GIS for Social Science Research},
            pdfauthor={Nicholas Thompson},
            pdfsubject={Mid-Term Exam},
            pdfborder={0 0 0},
            breaklinks=true}
\urlstyle{same}  % don't use monospace font for urls
\usepackage{color}
\usepackage{fancyvrb}
\newcommand{\VerbBar}{|}
\newcommand{\VERB}{\Verb[commandchars=\\\{\}]}
\DefineVerbatimEnvironment{Highlighting}{Verbatim}{commandchars=\\\{\}}
% Add ',fontsize=\small' for more characters per line
\usepackage{framed}
\definecolor{shadecolor}{RGB}{248,248,248}
\newenvironment{Shaded}{\begin{snugshade}}{\end{snugshade}}
\newcommand{\KeywordTok}[1]{\textcolor[rgb]{0.13,0.29,0.53}{\textbf{{#1}}}}
\newcommand{\DataTypeTok}[1]{\textcolor[rgb]{0.13,0.29,0.53}{{#1}}}
\newcommand{\DecValTok}[1]{\textcolor[rgb]{0.00,0.00,0.81}{{#1}}}
\newcommand{\BaseNTok}[1]{\textcolor[rgb]{0.00,0.00,0.81}{{#1}}}
\newcommand{\FloatTok}[1]{\textcolor[rgb]{0.00,0.00,0.81}{{#1}}}
\newcommand{\ConstantTok}[1]{\textcolor[rgb]{0.00,0.00,0.00}{{#1}}}
\newcommand{\CharTok}[1]{\textcolor[rgb]{0.31,0.60,0.02}{{#1}}}
\newcommand{\SpecialCharTok}[1]{\textcolor[rgb]{0.00,0.00,0.00}{{#1}}}
\newcommand{\StringTok}[1]{\textcolor[rgb]{0.31,0.60,0.02}{{#1}}}
\newcommand{\VerbatimStringTok}[1]{\textcolor[rgb]{0.31,0.60,0.02}{{#1}}}
\newcommand{\SpecialStringTok}[1]{\textcolor[rgb]{0.31,0.60,0.02}{{#1}}}
\newcommand{\ImportTok}[1]{{#1}}
\newcommand{\CommentTok}[1]{\textcolor[rgb]{0.56,0.35,0.01}{\textit{{#1}}}}
\newcommand{\DocumentationTok}[1]{\textcolor[rgb]{0.56,0.35,0.01}{\textbf{\textit{{#1}}}}}
\newcommand{\AnnotationTok}[1]{\textcolor[rgb]{0.56,0.35,0.01}{\textbf{\textit{{#1}}}}}
\newcommand{\CommentVarTok}[1]{\textcolor[rgb]{0.56,0.35,0.01}{\textbf{\textit{{#1}}}}}
\newcommand{\OtherTok}[1]{\textcolor[rgb]{0.56,0.35,0.01}{{#1}}}
\newcommand{\FunctionTok}[1]{\textcolor[rgb]{0.00,0.00,0.00}{{#1}}}
\newcommand{\VariableTok}[1]{\textcolor[rgb]{0.00,0.00,0.00}{{#1}}}
\newcommand{\ControlFlowTok}[1]{\textcolor[rgb]{0.13,0.29,0.53}{\textbf{{#1}}}}
\newcommand{\OperatorTok}[1]{\textcolor[rgb]{0.81,0.36,0.00}{\textbf{{#1}}}}
\newcommand{\BuiltInTok}[1]{{#1}}
\newcommand{\ExtensionTok}[1]{{#1}}
\newcommand{\PreprocessorTok}[1]{\textcolor[rgb]{0.56,0.35,0.01}{\textit{{#1}}}}
\newcommand{\AttributeTok}[1]{\textcolor[rgb]{0.77,0.63,0.00}{{#1}}}
\newcommand{\RegionMarkerTok}[1]{{#1}}
\newcommand{\InformationTok}[1]{\textcolor[rgb]{0.56,0.35,0.01}{\textbf{\textit{{#1}}}}}
\newcommand{\WarningTok}[1]{\textcolor[rgb]{0.56,0.35,0.01}{\textbf{\textit{{#1}}}}}
\newcommand{\AlertTok}[1]{\textcolor[rgb]{0.94,0.16,0.16}{{#1}}}
\newcommand{\ErrorTok}[1]{\textcolor[rgb]{0.64,0.00,0.00}{\textbf{{#1}}}}
\newcommand{\NormalTok}[1]{{#1}}
\usepackage{graphicx,grffile}
\makeatletter
\def\maxwidth{\ifdim\Gin@nat@width>\linewidth\linewidth\else\Gin@nat@width\fi}
\def\maxheight{\ifdim\Gin@nat@height>\textheight\textheight\else\Gin@nat@height\fi}
\makeatother
% Scale images if necessary, so that they will not overflow the page
% margins by default, and it is still possible to overwrite the defaults
% using explicit options in \includegraphics[width, height, ...]{}
\setkeys{Gin}{width=\maxwidth,height=\maxheight,keepaspectratio}
\setlength{\parindent}{0pt}
\setlength{\parskip}{6pt plus 2pt minus 1pt}
\setlength{\emergencystretch}{3em}  % prevent overfull lines
\providecommand{\tightlist}{%
  \setlength{\itemsep}{0pt}\setlength{\parskip}{0pt}}
\setcounter{secnumdepth}{0}

%%% Use protect on footnotes to avoid problems with footnotes in titles
\let\rmarkdownfootnote\footnote%
\def\footnote{\protect\rmarkdownfootnote}

%%% Change title format to be more compact
\usepackage{titling}

% Create subtitle command for use in maketitle
\newcommand{\subtitle}[1]{
  \posttitle{
    \begin{center}\large#1\end{center}
    }
}

\setlength{\droptitle}{-2em}
  \title{GVPT392(849): Introduction to GIS for Social Science Research}
  \pretitle{\vspace{\droptitle}\centering\huge}
  \posttitle{\par}
\subtitle{Mid-Term Exam}
  \author{Nicholas Thompson}
  \preauthor{\centering\large\emph}
  \postauthor{\par}
  \predate{\centering\large\emph}
  \postdate{\par}
  \date{9am Oct3 - 5pm Oct 7, 2016}


% Redefines (sub)paragraphs to behave more like sections
\ifx\paragraph\undefined\else
\let\oldparagraph\paragraph
\renewcommand{\paragraph}[1]{\oldparagraph{#1}\mbox{}}
\fi
\ifx\subparagraph\undefined\else
\let\oldsubparagraph\subparagraph
\renewcommand{\subparagraph}[1]{\oldsubparagraph{#1}\mbox{}}
\fi


\begin{document}
\maketitle

The following coverages can be found in the PACD8 folder.

\begin{enumerate}
\def\labelenumi{\arabic{enumi}.}
\item
  PA\_CD8\_Voterfile \(=\) all registered voters for Pennsylvania,
  Congressional District 8. This is north-suburban Philadelphia,
  including all of Bucks and part of Montgomery Counties.
\item
  PA\_CD8\_Boundary \(=\) the outline for CD 8.
\item
  PA\_and\_NJ\_Counties \(=\) County boundaries for the two states.
\item
  Four States = State boundaries for PA, DE, NJ, and MD.
\item
  CD8\_PA\_Pct\_Data\_2012 \(=\) voter precinct data for 2012.
\item
  Mont\_County\_Recent\_Movers\_10\_12.
\item
  Bucks\_County\_Recent\_Movers\_10\_12.
\item
  CD8\_Places.
\item
  PA CD8\_Tracts.
\end{enumerate}

Three files above contain points for voters at their residences. These
are 1, 6, and 7. For these files, the following columns contain
important information:

Age (and Year Born) \(=\) the age of the voter in 2012.

Rep\_Party, Dem\_Party, Ind\_Unaf\_Party \(=\) the party registration of
the voter: Rep \(=\) Republican, Dem \(=\) Democratic, and Ind\_Unaf
\(=\) Independent/Unaffiliated.

And there are other items that will be less important for this exercise.

\begin{center}\rule{0.5\linewidth}{\linethickness}\end{center}

For the following questions, use whatever tools you deem appropriate
form the ArcGIS package, but be sure to describe what you did to address
the questions. Be resourceful, but you need not write more than one page
in response to each question.

\begin{enumerate}
\def\labelenumi{\arabic{enumi}.}
\tightlist
\item
  Aggregate the voter and mover data to the census tract level for PA
  CD8.
\end{enumerate}

~~~~~~To aggregate the data, I used a three phase process with multiple
steps in each phase. In Phase 1, I imported the data using the catalogue
in ArcMap. To import the data I first created a geodatabase file named
\texttt{exam}. Here I imported all exam shapefiled included in the
provided exam folder by right clicking on the \texttt{exam.gdb} and
selecting import from multiple. Next I systemtaically added four file
layers to the ArcMap table of contents:

~~~~~~a. PA\_CD8\_Voterfile (hereafter depicted as \texttt{voter});

~~~~~~b. Mont\_County\_Recent\_Movers\_10\_12 (hereafter depicted as
\texttt{MC});

~~~~~~c. Buck\_County\_Recent\_Movers\_10\_12 (hereafter depicted as
\texttt{BC});

~~~~~~d. PA\_CD8\_Tracts (hereafter depicted as \texttt{tracts}).

This was the end of Phase 1.

~~~~~~In Phase 2, I reviewed the data and deleted unnecessary fields.
The number is too great to depict which were removed. I kept essential
fields outlined in the instructions above, as well as some others that I
anticipated would be necessary (including \texttt{MOVER} from the
\texttt{voter} file, \texttt{ozipcode} and \texttt{dzipcode} from
\texttt{MC} and \texttt{BC}, and \texttt{ORNIC}, \texttt{DRNIC}, and
\texttt{RNIC} from \texttt{voter}, \texttt{BC}, and \texttt{MC}). The
combination of fields chosen allowed me to manipulate the data to
achieve the desired results. I removed fields by double-clicking on each
layer in the table of contents and navigating to the \texttt{Fields}
tab. After clearing all of the fields, I was able to check only the
fields I wanted to keep. Next, I exported the data into new layers
within the geodatabase. This data management process ended Phase 2.

~~~~~~In Phase 3, I used the \texttt{Spatial\ Join} feature (hereafter
known as \texttt{SJ}) to systemtaically join the layers. First I
conducted a \texttt{SJ} of \texttt{voters} to \texttt{tracts} and
created a new layer called \texttt{tracts01}. Next I created the
following \texttt{SJ}s:

~~~~~~a. \texttt{tracts} \(+\) \texttt{BC} \(=\) \texttt{tracts02}

~~~~~~b. \texttt{tracts} \(+\) \texttt{MC} \(=\) \texttt{tracts04}

~~~~~~c. \texttt{tracts01} \(+\) \texttt{tracts02} \(=\)
\texttt{tracts03}

~~~~~~d. \texttt{tracts03} \(+\) \texttt{tracts04} \(=\)
\texttt{tracts07}

The last combination created a spatially joined dataset depicting the
north-suburban part of Philadelphia.

\begin{itemize}
\tightlist
\item
  Then compute and calculate the Democratic \(\%\) of total registered
  voters (10 points).
\end{itemize}

To compute and calculate the Democratic \(\%\) of total registered
voters I needed to create a new field in the \texttt{BC} and \texttt{MC}
shapefiles. I completed these computations prior to merging all of the
data to ensure that they were carried over in each of the \texttt{SJ}s.
First, I created a new field called \texttt{vote\_total}. Using the
field calculator tool, I added the \texttt{Republican},
\texttt{Democratic}, and \texttt{Independent} fields together. This
produced a one in each row of the \texttt{vote\_total}. Next, I used the
statistics tool to calculate the total sum of from the
\texttt{Democratic} field and the the sum from the \texttt{vote\_total}
fields. I conducted statistical analysis before and after conducting the
joins. The Table 1 below shows the outcomes. Note there is no
significant difference in the percentages either pre- or post-join.

\begin{table}[]
\centering
\caption{Percentage of Democratic Voters}
\begin{tabular}{ccc}
Field       & Pre-Join & Post-Join \\
\hline
Democratic  & 71,048   & 133,467   \\
vote\_total & 540,451  & 1,019,887 \\
\hline
\hline
Percentages & 53.23 \% & 52.99 \% 
\end{tabular}
\end{table}

\begin{itemize}
\tightlist
\item
  Compute and calculate the Democratic \(\%\) of total movers in Bucks
  and Montgomery counties (10 points).
\end{itemize}

To calculate the percentage of democratic movers I created two fields in
\texttt{BC} and \texttt{MC} once callde \texttt{zip\_dif} and another
called \texttt{move2}. The \texttt{zip\_dif} field captured a difference
between the originating zip code (\texttt{ozipcode}) of each voter in
the respective counties and the destination zip code
(\texttt{dzipcode}). Next, a phython code converted the
\texttt{zip\_dif} field into a \(1\) or a \(0\). This allowed me to
total the number of people that moved from one zip code to another.
Table 2 shows the results of this computation.

\begin{Shaded}
\begin{Highlighting}[]
\NormalTok{def }\KeywordTok{is_positive}\NormalTok{(x):}
\StringTok{  }\NormalTok{if (}\KeywordTok{abs}\NormalTok{(x)>}\DecValTok{0}\NormalTok{):}
\StringTok{    }\NormalTok{return }\DecValTok{1}
  \KeywordTok{elif} \NormalTok{(}\KeywordTok{abs}\NormalTok{(x)==}\DecValTok{0}\NormalTok{):}
\StringTok{    }\NormalTok{return }\DecValTok{0} 
\end{Highlighting}
\end{Shaded}

\begin{table}[]
\centering
\caption{Percentage of Democratic Movers}
\label{my-label}
\begin{tabular}{cccc}
Field       & Montgomery & Bucks    & Sum      \\
\hline
Democratic  & 2,016      & 22,555   & 24,571   \\
move\_total & 4,620      & 42,040   & 46,660   \\
\hline
\hline
Percentages & 43.63 \%   & 53.65 \% & 52.66 \%
\end{tabular}
\end{table}

\begin{itemize}
\tightlist
\item
  Produce two maps of these percentages.
\end{itemize}

To produce my maps I followed some formatting guidelines. First, I
always included a legend, scale, and north seeking arrow. Also included
was my name as the author and the date I finalized the map. For this
first set of maps I normalized the \texttt{Sum\_Sum\_Democratic}
variable over the \texttt{Sum\_Sum\_vote\_total} for the first map (as
labeled in Figure 1). I normalized the \texttt{Sum\_Sum\_Democratic}
variable over the \texttt{Sum\_Sum\_vote\_total} variable for the second
requirement.

This depicts a strong concentration of Democratic voters located in the
southeastern portion of the country. Democratic voters show a propensity
for migration with a large percentage moving in and around the
southeastern portion of the county.

Note that this data does not reflect the \texttt{MOVER} field from the
\texttt{MC} dataset. The \texttt{MOVER} variable was not used because it
was only available in the \texttt{MC} dataset and only depicted a small
swath of migration running from northwest to southeast along the
southwestern third of the county. All map figures will be available in
the Appendix.

\begin{enumerate}
\def\labelenumi{\arabic{enumi}.}
\setcounter{enumi}{1}
\tightlist
\item
  How would you characterize the spatial distribution of Republicans,
  Democrats, and Independents in PA CD8? Write up two paragraphs based
  on what you have found, describing how you used ArcGIS to address the
  question. (20 points)
\end{enumerate}

Democrats, by and large, tend to be located in the south eastern part of
the map (which is northern Philadelphia). Republicans are a
significantly lesser amount of the population and tend to inhabit the
norther part of the voting district, with a concentration in the center
of Montgomery county in the southwest. Independents show similar
patterns to the Republicans but on a much smaller scale.

I used a percentage of total voters for each party and compared them to
one another, as depicted in Figure 2. This distribution by percentage
was calculated using the \texttt{Sum\_Sum\_Democrat},
\texttt{Sum\_Sum\_Republican} and \texttt{Sum\_Sum\_Independent}
variables normalized by \texttt{Sum\_Sum\_vote\_total} variable. I
utilized a five quantile break.

\begin{enumerate}
\def\labelenumi{\arabic{enumi}.}
\setcounter{enumi}{2}
\tightlist
\item
  The data included also show two populations of recent movers from
  inside PA and from nearby states. How do the recent movers in
  Montogomery and Bucks counties compare by age and by party
  registration to the entire PA\_CD8 voting populations? Explain your
  answer in no more than one page.
\end{enumerate}

To answer this question I pulled the descriptive statistics for each
dataset (\texttt{BC}, \texttt{MC}, \texttt{voter}) and collected the
number of people that voted in each party, the party means, and the
party standard deviations (Table 3). I also collected the mean, standard
deviation, minimum, and maximum values of the voters from each of the
three datasets (Table 4).

The Montgomery county movers are represented by a higher percentage of
Republicans than Democrats (Independents are behing Republicans and
Democrats in all three datasets. The Bucks county Democratic movers
(\(46.10\%\)) have a significantly higher percentage than Montgomery
county movers (\(37.68\%\)). This indicates that either Democrats in
Bucks county are migrating at higher rates, or that the percentage of
voters in both counties falls around those means.

One way to differentiate between the two possibilities is to compare the
means of the total voter population from the district. If we assume a
normal distribution of Republican, Democratic, and Independent voters
throughout the district then the averages of \(34.13\%\) for
Republicans, \(52.63\%\) for Democratic party voters, and \(12.10\%\)
for Independent voters should represent similar means the two counties.

The Republican numbers in Bucks county are close to the overall voter
mean, but the Montgomery mean is higher. This indicates that Republicans
are migrating to Montgomery county. The mean of Democratic voters in
both Bucks and Montgomery counties is significantly lower than in the
\texttt{voter} database. This may indicate less migration for Democratic
voters.

The age demographics do not tell us as much without a geographic
dispersion. The mean and standard deviation of the three data sets is
relatively similar, as depicted in Table 4.

\begin{table}[]
\centering
\caption{Descriptive Statistics by Party}
\label{my-label}
\begin{tabular}{ccccc}
Dataset    & Party       & Vote Totals & Mean   & Std\_Dev \\ \hline
Montgomery & Republican  & 2230         & 0.4167 & 0.4930    \\
Montgomery & Democratic  & 2016         & 0.3768 & 0.4845   \\
Montgomery & Independent & 1043         & 0.1949 & 0.3962   \\ \hline
Bucks      & Republican  & 16808        & 0.3436 & 0.4748   \\
Bucks      & Democratic  & 22555        & 0.461  & 0.4984   \\
Bucks      & Independent & 8979         & 0.1835 & 0.3871   \\ \hline
Voter      & Republican  & 46085        & 0.3413 & 0.4742   \\
Voter      & Democratic  & 71048        & 0.5263 & 0.4993   \\
Voter      & Independent & 16334        & 0.121  & 0.3261  \\ \hline \hline
\end{tabular}
\end{table}

\begin{table}[]
\centering
\caption{Descriptive Statistics by Age}
\label{my-label}
\begin{tabular}{lllll}
\multicolumn{1}{c}{Data Set} & \multicolumn{1}{c}{Mean} & \multicolumn{1}{c}{\begin{tabular}[c]{@{}c@{}}Standard \\ Deviation\end{tabular}} & \multicolumn{1}{c}{Min} & \multicolumn{1}{c}{Max} \\ \hline
Montgomery                   & 43.81                    & 16.73                                                                             & 20                      & 104                     \\
Bucks                        & 44.83                    & 15.99                                                                             & 20                      & 104                     \\
Voter Data                   & 49.57                    & 17.50                                                                             & 0                       & 112                    \\ 
\hline 
\hline
\end{tabular}
\end{table}

\section{Appendix}\label{appendix}

\begin{figure}[htbp]
\centering
\includegraphics{question_1-2.png}
\caption{Figure 1: Democratic Voters and Movers}
\end{figure}

\begin{figure}[htbp]
\centering
\includegraphics{question_2.png}
\caption{Figure 2: Parties in Space}
\end{figure}

\end{document}
